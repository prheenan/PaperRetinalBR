
\documentclass{article}

\usepackage[hyperref,doi=false,url=false,backref,style=alphabetic,maxbibnames=99,backend=bibtex,citestyle=authoryear]{biblatex}
\usepackage{gensymb} % for Celsius
\usepackage{amsmath}
\usepackage{amsthm}
\usepackage{amssymb} 
\usepackage{braket}
\usepackage{geometry}
\geometry{margin=1in}

% graphics...
\usepackage{graphicx}
\usepackage{fancyhdr} % Custom headers and footers
\pagestyle{fancyplain} % Makes all pages in the document conform to the custom headers and footers
\fancyhead{} % No page header - if you want one, create it in the same way as the footers below
\fancyfoot[L]{} % Empty left footer
\fancyfoot[C]{} % Empty center footer
\fancyfoot[R]{\thepage} % Page numbering for right footer
\renewcommand{\headrulewidth}{0pt} % Remove header underlines
\renewcommand{\footrulewidth}{0pt} % Remove footer underlines
\setlength{\headheight}{13.6pt} % Customize the height of the header

\setlength\parindent{0pt} % Removes all indentation from paragraphs - comment this line for an assignment with lots of text

%----------------------------------------------------------------------------------------
%	TITLE SECTION
%----------------------------------------------------------------------------------------

\newcommand{\ndiffn}[3]{\frac{\partial^{#3}#1}{\partial^{#3}#2}}
\newcommand{\diffn}[2]{\ndiffn{#1}{#2}{}}
\newcommand{\eqs}[1]{
\begin{align*} 
\begin{split}
#1
\end{split}					
\end{align*}}

\newcommand{\pFig}[3]{
\begin{figure}[h!]
\centering
\caption{#2}
\boxed{\includegraphics[width=0.75\textwidth]{#1}}
\label{figure:#3}
\end{figure}}



\newcommand{\eqlab}[2]{
\begin{equation}
\label{equation:#2}
\begin{split}
#1
\end{split}
\end{equation}
}


\newcommand{\brac}[1]
{
  \Big[ #1 \Big]
}

\newcommand{\tab}[2]
{
\begin{tabular}[h!]{#1}
\hline
#2
\end{tabular}
}

\newcommand{\e}[0]{\\ \hline}

\newcommand{\pMat}[1]{
 \left[ \begin{array}
#1 \end{array} \right]
}

\newcommand{\code}[1]{
  \begin{lstlisting}[language=Python]
#1
\end{lstlisting}
}

% for making specific references.
\newcommand{\tRef}[1]{Table \ref{table:#1}}
\newcommand{\fRef}[1]{Figure \ref{figure:#1}}
\newcommand{\sRef}[1]{Section \ref{section:#1}}
\newcommand{\eRef}[1]{Equation \ref{equation:#1}}

\newcommand{\pLit}[4]{
\cite{#1}
\begin{adjustwidth}{2.5em}{0pt}
\textbf{Title}: \citetitle{#1} \\
\textbf{Keywords}: #4 \\
\textbf{Big Picture}: #2 \\
\textbf{Summary}: #3 
\end{adjustwidth}
}



\title{}
\author{Patrick Heenan} % Your name

\date{\normalsize\today} % Today's date or a custom date

\begin{document}

\maketitle

Our current BR/BO data measures the total work as a function of cantilever base position z:

\eqlab{W_{\text{total}}(z) = W_{\text{PEG}}(z) + W_{\text{probe}}(z) + W_{\text{protein}}(z) }{tot}

However, we are interested in the system work (this is essentially what we meaured for the JCP paper, in the absence of PEG3400)

\eqlab{W_{\text{system}}(z) = W_{\text{probe}}(z) + W_{\text{protein}}(z) }{sys}

Note, the probe work is included as an input to Jarzynski, but we eventually remove it using and IWT or WHAM (not discussed here). By Jarzynski, we have:

\eqlab{ 
\exp(-\beta A_{\text{total}}(z))&= \braket{\exp(-\beta W_{\text{total}}(z))}_N \\
\exp(-\beta A_{\text{PEG}}(z)) &=  \braket{\exp(-\beta W_{\text{PEG}}(z))}_N\\
}{common}

where $\beta=\frac{1}{k_\text{B} T}$, $\braket{W(z)}_N$ denotes the enesemble average of the works at a given z. Dividing the two equations above gives (making z-dependence implicit for notational simplicity):

\eqlab{ \exp(-\beta (A_{\text{total}}-A_{\text{PEG}})) =
\frac{\braket{\exp(-\beta W_{\text{total}})}_N }
{\braket{\exp(-\beta W_{\text{PEG}})}_N} = \exp(-\beta A_{\text{system}}) 
 }{diff}

Where in the last step we used that $A_{\text{total}}-A_{\text{system}} = A_{\text{PEG}}$, by \eRef{tot} and \eRef{sys}. We also know by Jarzynski

\eqs{ \exp(-\beta (A_{\text{total}}-A_{\text{PEG}})) 
&=\braket{\exp(-\beta (W_{\text{total}} - W_{PEG}))}_N \\
&= \braket{\exp(-\beta  ((W_{\text{PEG}} + W_{\text{probe}} + W_{\text{protein}}) - W_{PEG}))}_N   \\
&= \braket{\exp(-\beta  (W_{\text{probe}} + W_{\text{protein}}))}_N \\
&= \exp(-\beta A_{\text{system}}) }

Where the last step used \eRef{diff}. In other words

\eqs{  \exp(-\beta A_{\text{system}}) =  
\boxed{\exp(-\beta  (A_{\text{probe}}(z) + A_{\text{protein}}) ) = 
\braket{\exp(-\beta  (W_{\text{probe}} + W_{\text{protein}}))}_N } \qed}









%----------------------------------------------------------------------------------------

\end{document}
